\documentclass{article}

\usepackage[mathbf]{euler}
\usepackage{longtable}
\usepackage{geometry}
\geometry{hmargin=0.5in,vmargin=1.2in}

%% Set up indexing headers and footer for fast lookup while
%% flipping pages.
\usepackage{fancyhdr}
\pagestyle{fancy}
\renewcommand{\headrulewidth}{0pt}
\setlength{\headheight}{14.0pt}
\lhead{\large\mathversion{bold}$\firstmark$}
\rhead{\large\mathversion{bold}$\botmark$}

\begin{document}

%% This function is a utility to help the sprintf
%% function keep the number of digits after the decimal
%% place at the right level to come out to 9 digits in total
%% both before and after the decimal point.
{:  def digs(x)
      if ("%0.8f" % x) =~ /[-+]?([0-9]*)\./
        l = 10 - $1.length  #$ This comment keeps syntax highlighting in emacs
                            # from getting confused.  The single dollar
                            # sign makes it think is in math mode.
                            # Occasionally needed when ruby and LaTeX mix.
        l = (l > 0 ? l : 0)
        return l
      end
      return(0)
    end
:}

%% Set up the longtable environment by specifying the header for
%% each page and the footer.
\begin{longtable}[c]{l|llllll}
\hline\hline
\multicolumn{1}{c|}{\mathversion{bold}$x$}&
\multicolumn{1}{c}{\mathversion{bold}$\sin(x)$}&
\multicolumn{1}{c}{\mathversion{bold}$\cos(x)$}&
\multicolumn{1}{c}{\mathversion{bold}$\tan(x)$}&
\multicolumn{1}{c}{\mathversion{bold}$\cot(x)$}&
\multicolumn{1}{c}{\mathversion{bold}$\sec(x)$}&
\multicolumn{1}{c}{\mathversion{bold}$\csc(x)$}\\
\hline\hline
\endhead
\hline\hline
\endfoot

{:
## Set up variables needed for computing the values that go into
## the table, taking care to use ruby's BigDecimal class to ensure
## accuracy.  The table is set up here to go from 0 to pi/2 in
## increments of step

   require 'bigdecimal'
   step = BigDecimal("0.001")
   one = BigDecimal("1.0")
   z0 = BigDecimal("0.0")
   p2 = BigDecimal((Math::PI / 2.0).to_s)
-:}

%% Treat the first line of the table, for an x value of 0 specially
%% by giving exact answers in symbolic form.
{:= "\\multicolumn{1}{c|}{{\\mathversion{bold}\\mark{0}$0$}}" :}&
{:= "\\multicolumn{1}{c}{{$0$}}" :}&
{:= "\\multicolumn{1}{c}{{$1$}}" :}&
{:= "\\multicolumn{1}{c}{{$0$}}" :}&
{:= "\\multicolumn{1}{c}{{$\\infty$}}" :}&
{:= "\\multicolumn{1}{c}{{$1$}}" :}&
{:= "\\multicolumn{1}{c}{{$\\infty$}}" :}\\

%% Here is the loop for the body of the table, which starts with
%% x one step beyond 0 and pre-computes some of the functions
{:   x = z0 + step
     while (x < p2) do
      tanx = BigDecimal(Math.tan(x).to_s)
      cotx = one / BigDecimal(Math.tan(x).to_s)
      secx = one / BigDecimal(Math.cos(x).to_s)
      cscx = one / BigDecimal(Math.sin(x).to_s)
:}

{:
## Here is where each line of the main body of the table is set.
## We use the digs function defined above to make sure that every
## column as the same number of significant digits for the functions
## that tend toward infinity.
##
## NB: I could write this comment as a LaTeX comment outside a code
## segment, but then it would end up in the generated LaTeX file,
## TrigTable.etx.  Doing so made that file about 6MB in size; by
## putting the comments inside this ruby code block, the size was
## around 2MB.
-:}
{: xf = "%0.4f" % x :}
\mark{{:= xf :}}
{:= "\\mathversion{bold}$%0.4f$" % x :}&
{:= "$%0.8f$" % Math.sin(x) :}&
{:= "$%0.8f$" % Math.cos(x) :}&
{:= "$%0.*f$" % [ digs(tanx), tanx ] :}&
{:= "$%0.*f$" % [ digs(cotx), cotx ] :}&
{:= "$%0.*f$" % [ digs(secx), secx ] :}&
{:= "$%0.*f$" % [ digs(cscx), cscx ] :}\\

%% Step and loop
{:
      x += step
    end
:}

%% Finally, treat the last row of the table for pi/2 specially
%% by giving exact values symbolically.
{:= "\\multicolumn{1}{c}{{\\mark{\\pi/{2}}\\mathversion{bold}$\\pi/{2}$}}" :}&
{:= "\\multicolumn{1}{c}{$1$}" :}&
{:= "\\multicolumn{1}{c}{$0$}" :}&
{:= "\\multicolumn{1}{c}{$\\infty$}" :}&
{:= "\\multicolumn{1}{c}{$0$}" :}&
{:= "\\multicolumn{1}{c}{$\\infty$}" :}&
{:= "\\multicolumn{1}{c}{$1$}" :}\\

%% End the table and document---this version comes to 315 pages!
\end{longtable}
\end{document}