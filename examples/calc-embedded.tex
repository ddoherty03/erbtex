\documentclass[11pt]{article}
\usepackage{calc-embedded}

%% Calc Embeded Mode Notes:
%% - C-x * a will try to activate all formulas found in the buffer
%% - C-x * e will activate a single formula
%% - C-x * w will activate a single word as a formula
%% - C-x * j will activate the formula then perform a select on the part
%%           of the formula that the cursor is on
%% - C-x * u will update the current formula; with C-u prefix, it
%%           updates the region; with a numeric prefix (say M-1), it
%%           updates the whole buffer.  "Updating" means that all =>
%%           formulas affected by the edit are recalculated.
%% - C-x * d will duplicate the current formula then activate it.  This
%%           is a good way to perform serial calculations on a series of formulas
%% - C-x * n moves to the next activated formula
%% - C-x * p moves to the prior activated formula; these are a good way
%%           to see what formulas are currently activated
%% - C-x * ` will edit the current formula in a separate buffer.  This
%%           is often easier than doing selects under calc with the j command
%%           just to make some non-mathematical changes.  It also
%%           updates any dependent => formulas on exit (C-c C-c)

%% You can get emacs to activate all formulas in the file on
%% loading by adding the local variable setting shown at the
%% bottom of this file to your file.  It will ask you to
%% confirm unless you customize the safe-local-eval-forms variable to
%% add the value (calc-embedded-activate) to the safe list.

%% Use m C in embedded mode to toggle automatic recalculation on and off.
%% Use m R in embedded mode to cycle through different ways to record modes.


\renewcommand{\frac}[2]{{\scriptstyle #1}/\!{\scriptstyle #2}}

\begin{document}
\( %%% 246:49 %%%
\frac{246}{49} \)

%%% 143. %%%
\( 143. \)

$ %%% a := 28 %%%
a \gets 28 $

$\evalto a! \to 304888344611713860501504000000 $

Here is a matrix:
\begin{displaymath}
%%% M := [[-1, 103, 16], [6, 5, 8], [0, 2, 1]] %%%
M \gets \begin{pmatrix} -1 & 103 & 6 \\ 6 & 5 & 8 \\ 0 & 2 & 1 \end{pmatrix}
\end{displaymath}

Here is its inverse:
% [calc-mode: fractions: t]
\begin{displaymath}
\evalto \frac{1}{M} \to \begin{pmatrix} \frac{11}{415} & 
 \frac{71}{415} & \frac{-744}{415} \\ \frac{6}{415} & 
 \frac{1}{415} & \frac{-104}{415} \\ \frac{-12}{415} & 
 \frac{-2}{415} & \frac{623}{415} \end{pmatrix}
\end{displaymath}

Here is an idea: $\int_0^\pi e^{-x^2} dx$, why not try that?

\end{document}

%%% Local Variables:
%%% eval:(calc-embedded-activate)
%%% End:
% [calc-global-mode: language: latex]
% [calc-global-mode: float-format: (float 0)]
% [calc-global-mode: angles: rad]
